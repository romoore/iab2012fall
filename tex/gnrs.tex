\documentclass[conference, 11pt]{IEEEtran}
\usepackage{iabproject}
\usepackage{times}
\usepackage{graphicx}
\usepackage{amsmath, amssymb,latexsym}
\usepackage{url}


%Change the date in \wintitle with the otpinal argument
\wintitle[Fall 2012]{Global Name Resolution Service}

%The optional argument for \winstudent is for the number of students
%If the project has only one student, omit the optional argument.
\winstudent[2]{Robert Moore\\
\email{romoore@cs.rutgers.edu}\\
Feixiong Zhang\\
\email{feixiong@winlab.rutgers.edu}}

%The optional argument for \makeadvisor is for the number of advisors
%If the project has only one advisor, omit the optional argument.
\winadvisor[4]{Richard P. Martin\\
\email{rmartin@cs.rutgers.edu}\\
Yanyong Zhang\\
\email{yyzhang@winlab.rutgers.edu}\\
Thu D. Nguyen\\
\email{tdnguyen@cs.rutgers.edu}\\
Kiran Nagaraja\\
\email{nkiran@winlab.rutgers.edu}}


\begin{document}
\maketitle
%%%%%%%%%%%%%%%%%%%
\begin{abstract}
%%%%%%%%%%%%%%%%%%%
The Global Name Resolution Service (GNRS) is a critical portion of the
Mobility First (MF) Future Internet Architecture (FIA). Providing sub-second
insert and retrieval of Globally Unique IDentifier (GUIDs) bindings enables
network support of highly mobile devices, content, and information contexts.
% TODO: This is placeholding junk.
blah, blah, blah, FIa is great and GNRS is important. We made it better by
throwing away a bunch of junk and making stuff more modular. Expand here and
make the words good.
%%%%%%%%%%%%%%%%%%%
\end{abstract}
%%%%%%%%%%%%%%%%%%%
\section{Introduction}
GNRS provides a distributed, redundant storage network for GUID $\rightarrow$
Network Address (NA) binding values.  By utilizing the latest secure hashing
algorithms and replicating across multiple autonomous systems, we are able to
scale the system to tens of thousands of nodes.  

The server provides the basic functionality of the GNRS, being both the point
of access and the storage node for the system.  Clients contact one or more
``access'' servers (configured during network attachment) to insert bindings
for their own GUID values, or to retrieve bindings for the GUID values found
in application contexts.
\subsection{Our Contribution}
As a research prototype, the GNRS server is written to be modular, extensible,
and configurable, and yet retains good real-world performance.  Simple
software interfaces connect the different components, allowing each to be
replaced or reconfigured independently of one another.
\subsection{Related Work}
The previous version of the server was a patchwork of last-minute rush jobs
and ``one-off'' contributions intended to be temporary, but which resulted in
an discordant mish-mosh of spaghetti code.  Due to a combination of unexpected
side-effects and undocumented functionality, the code had to be refactored.
\section{Technical Approach}

\subsection{A Important Subsection}
Important
\section{Summary of Results}
Summary
\section{Project Status}
Status

A new project might begin this section with  text like
\begin{quotation}
\em This project was started in Fall 2006. Based on our preliminary
results, we will investigate \ldots
\end{quotation} 

\bibliographystyle{unsrt}
%\bibliography{gnrs}

\end{document}
